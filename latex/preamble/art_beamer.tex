%% %%%%%%%%%%%%%%%%%%%%%%%%%%%%%%%%%%%%%%%%%%%%%%%%%%%%%%%%%%
%% art.tex
%% Layout, Font, erg. Pakete etc
%% Stand 2022/04/20
%% %%%%%%%%%%%%%%%%%%%%%%%%%%%%%%%%%%%%%%%%%%%%%%%%%%%%%%%%%%

\usepackage{onlyamsmath}			% Fehler bei Eingabe Mathematik
\usepackage[l2tabu,orthodox]{nag} 	% Fehler im LaTeX-Code


%% -- Für alle, die kein UTF8 eingestellt haben
%% --

\usepackage{selinput} 		% Automatische Wahl der "encoding"
	\SelectInputMappings{	% für alle, die immer noch kein UTF8 nutzen	
		,adieresis={ä}		% texdoc selinput
       	,germandbls={ß}		%
            }
\usepackage[T1]{fontenc}   % 


%% --  Wir schreiben deutschen Text und nutzen die deutschen Trennungsregeln
%% --

\usepackage[english]{babel}
% \usepackage[babel,german=guillemets]{csquotes} % \enquote{Text}
% \babelprovide[hyphenrules=ngerman-x-latest]{ngerman}


%% -- Times New Roman
%% -- das möchten wir nicht benutzen

%\usepackage{mathptmx}			
%	\usepackage[scaled=.90]{helvet}
%	\usepackage{courier}
	
	
%% -- Alternative lmodern
%% -- das wollen wir

\usepackage{lmodern}
\DeclareMathVersion{sans}
	\SetSymbolFont{letters}{sans}{OML}{cmbr}{m}{it} % Math letters from Latin Modern Sans
	\SetSymbolFont{operators}{sans}{OT1}{lmss}{m}{n} % Math operators
	\SetSymbolFont{symbols}{sans}{OMS}{lmsy}{m}{n} % Math symbols
	\SetMathAlphabet{\mathrm}{sans}{OT1}{lmr}{m}{n} % Large symbols
	\SetMathAlphabet{\mathsf}{sans}{OT1}{lmss}{m}{n}
	\SetMathAlphabet{\mathit}{sans}{OT1}{lmr}{m}{it}

%% --


%% -- Layout siehe KOMA-Script

% \addtokomafont{title}{\let\huge\Large} 
% \addtokomafont{author}{\large} 
% \addtokomafont{date}{\normalsize}
%
% \renewcommand*{\thesubsection}{\arabic{subsection}.\kern1pt}
% \RedeclareSectionCommand[font = \itshape]{subsection}
% \setcounter{secnumdepth}{\subsubsectionnumdepth} 
%
% \renewcommand*{\thesubsubsection}{\arabic{subsubsection}.\kern2pt} 
%
% \RedeclareSectionCommand[%
%  	,beforeskip		= .5\baselineskip 	% Etwas Abstand
%	,afterskip		= -0.1em			% Spitzmarke, d.h. kein Absatz
%	,font			= \normalfont	%	
%	]{subsubsection}


%% -- Sinnvolle Pakete
%% --
%% -- Bessere Listen 

\usepackage[shortlabels,inline]{enumitem} 	% texdoc enumitem

%% -- Große Tabellen

\usepackage{longtable}

%% -- Kommentare

\usepackage{verbatim}


\usepackage[singlespacing]{setspace}
\usepackage[onehalfspacing]{setspace}
% \usepackage[doublespacing]{setspace}

\usepackage{tikz}
\usetikzlibrary{tikzmark,fit}

%% -- Kopf/Fußzeile
%% -- 
% \usepackage{scrlayer-scrpage}
% \clearpairofpagestyles
% \addtokomafont{pagehead}{\upshape}
% \ohead{\footnotesize\authorss{}}
% \cfoot{\pagemark}
% \ihead{\footnotesize\numbering}


%% -- Weitere sinnvolle Pakete
%% --
\usepackage[%
	,newcommands
	,footnotes
	,raggedrightboxes
		]{ragged2e}	% Für Flattersatz
	
\usepackage{graphicx}

%% --  Mathematikumgebungen

\usepackage{amsmath,amsthm,amssymb}
\usepackage{mathtools}


%% -- Mathematische Umgebungen
%% -- Siehe amsthm-manual

% \theoremstyle{plain}
%
% \newtheorem{theorem}{Theorem}[section]
% \newtheorem{thm}[theorem]{Theorem}
% \newtheorem{proposition}[theorem]{Satz}
% \newtheorem{prop}[theorem]{Satz}
% \newtheorem{satz}[theorem]{Satz}
% \newtheorem{corollary}[theorem]{Korollar}
% \newtheorem{cor}[theorem]{Korollar}
% \newtheorem{korollar}[theorem]{Korollar}
% \newtheorem{lemma}[theorem]{Lemma}
% \newtheorem{lem}[theorem]{Lemma}
%
% \theoremstyle{definition}
%
% \newtheorem{definition}{Definition}[section]
% \newtheorem{defn}[definition]{Definition}
% \newtheorem{example}{Beispiel}[section]
% \newtheorem{examp}[example]{Beispiel}
% \newtheorem{beispiel}[example]{Beispiel}
%
% \theoremstyle{remark}
%
% \newtheorem*{note}{Anmerkung}
% \newtheorem*{rem}{Anmerkung}
% \newtheorem*{remark}{Anmerkung}


%% -- Richtige Abkürzungen; bitte auch in xspace-manual reinsehen
%% -- Bei Verwendung von \usepackage{hyperref} nach diesem Paket laden
\usepackage{xspace}
\newcommand{\zB}{z.\,B.\xspace}
\newcommand{\og}{o.\,g.\xspace}
\newcommand{\Dh}{d.\,h.\xspace}
\newcommand{\etc}{etc.\xspace}
\newcommand{\bzw}{bzw.\xspace}


%% -- Querverweise Pakete
% \usepackage[ngerman]{varioref}
% \usepackage[breaklinks = true]{hyperref}
% \usepackage{cleveref}
% \hypersetup{
% 	,colorlinks = true % Für Onlineversion true
% 	,urlcolor = blue % Farbe
% 	,citecolor = blue %
% 	,linkcolor = blue %
% 	% ,hidelinks % Vor dem Druck entfernen
% }


%% -- schwarzes Quadrat
\renewcommand{\qedsymbol}{$\blacksquare$}


%% -- Aligned Overset
\usepackage{aligned-overset}


\usepackage{tabularx}

\AtBeginSection[]{
  \begin{frame}
  \vfill
  \centering
  \begin{beamercolorbox}[sep=8pt,center,shadow=true,rounded=true]{title}
    \usebeamerfont{title}\insertsectionhead\par%
  \end{beamercolorbox}
  \vfill
  \end{frame}
}


\allowdisplaybreaks

\usepackage{svg}