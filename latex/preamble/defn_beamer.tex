%% %%%%%%%%%%%%%%%%%%%%%%%%%%%%%%%%%%%%%%%%%%%%%%%%%%%%%%%%%%
%% Einige sinnvolle Definitionen
%% und nützliche Pakete
%% Eigene Ergäzungen in My-Def.tex
%% Stand: 2022/04/20
%% %%%%%%%%%%%%%%%%%%%%%%%%%%%%%%%%%%%%%%%%%%%%%%%%%%%%%%%%%%

%% -- Zahlen
\newcommand{\N}{\mathbb{N}}		% Natürliche Zahlen
\newcommand{\Z}{\mathbb{Z}}		% Ganze Zahlen
\newcommand{\Q}{\mathbb{Q}}		% Rationale Zahlen
\newcommand{\R}{\mathbb{R}}		% Reelle Zahlen
\newcommand{\C}{\mathbb{C}}		% Komplexe Zahlen
% \newcommand{\K}{\mathbb{K}}		% Körperzeichen
\newcommand{\F}{\mathbb{F}}		% Galoiskörper
% \usepackage[sans]{dsfont}       % für die doppelte 1
\usepackage{dsfont}


%% -- Norm, Absolutbetrag, duales Paar siehe mathtools Abschnitt 3.6
\DeclarePairedDelimiterX{\norm}[1]{\lVert}{\rVert}{\ifblank{#1}{\:\cdot\:}{#1}}
\DeclarePairedDelimiterX{\abs}[1]{\lvert}{\rvert}{\ifblank{#1}{\:\cdot\:}{#1}}
\DeclarePairedDelimiterX{\dualp}[2]{\langle}{\rangle}{\ifblank{#1#2}{\,\cdot\,,\cdot\,}{\,#1,#2\,}}


%% -- Intervalle 
\newcommand{\interval}[1]{\left[ #1 \right]}	
\newcommand{\ointerval}[1]{\left] #1 \right[}	
\newcommand{\rointerval}[1]{\left[ #1 \right[}
\newcommand{\lointerval}[1]{\left] #1 \right]}


%% -- Ableitungen
%% -- auch als Beispiel für mathematische Ausdrücke
\newcommand*{\ds}{\mathop{}\!\mathrm{d}{s}}         % \ds = ds, 
\newcommand*{\dt}{\mathop{}\!\mathrm{d}{t}}         % \dt = dt, 
\newcommand*{\dx}{\mathop{}\!\mathrm{d}{x}}         % \dx = dx,
\newcommand*{\dy}{\mathop{}\!\mathrm{d}{y}}         % \dy = dy,
\newcommand*{\dP}{\mathop{}\!\mathrm{d}{\mathbb{P}}}         % \dy = dy,
\newcommand*{\diff}[1]{\mathop{}\!\mathrm{d}{#1}}	% \diff{\mu} = d\mu, 
\newcommand*{\dL}[1][1]{\mathop{}\!\mathrm{d}{\mathcal{L}^{#1}}}
                                                    % \dL = dL^1



%% var-Symbole anstelle der Originalen, besser zu unterscheiden
%% --
\let\ORGvarepsilon=\varepsilon
\let\varepsilon=\epsilon
\let\epsilon=\ORGvarepsilon
%
\let\ORGvarrho=\varrho
\let\varrho=\rho
\let\rho=\ORGvarrho
%
\let\ORGvartheta=\vartheta
\let\vartheta=\theta
\let\theta=\ORGvartheta
%
\let\ORGvarphi=\varphi
\let\varphi=\phi
\let\phi=\ORGvarphi
%
\let\ORGvarleq=\leqslant
\let\leqslant=\leq
\let\leq=\ORGvarleq
%
\let\ORGvargeq=\geqslant
\let\geqslant=\geq
\let\geq=\ORGvargeq


%% -- Träger
\newcommand{\supp}{\mathrm{supp}}


%% -- abzählbare Summen, Vereinigungen, etc.
\newcommand{\sumk}[1][1]{\sum_{k=#1}^{\infty}}
\newcommand{\suml}[1][1]{\sum_{l=#1}^{\infty}}
\newcommand{\cupk}[1][1]{\bigcup_{k=#1}^{\infty}}
\newcommand{\cupl}[1][1]{\bigcup_{l=#1}^{\infty}}
\newcommand{\limk}{\lim_{k \to \infty}}
\newcommand{\limn}{\lim_{n \to \infty}}


%% -- komplexe Zahlen
\newcommand{\cre}[1]{\mathrm{Re}(#1)}
\newcommand{\cim}[1]{\mathrm{Im}(#1)}
\newcommand{\I}{\mathrm{i}}

%% -- stochastische Konvergenzen
\newcommand{\tod}{\overset{d}{\to}}
\newcommand{\tov}{\overset{v}{\to}}

\newcommand{\todn}{\underset{n \to \infty}{\overset{d}{\longrightarrow}}}
\newcommand{\tovn}{\underset{n \to \infty}{\overset{v}{\longrightarrow}}}


%% --
% Menge der reellwertigen stetigen Abbildungen mit kompaktem Träger
\newcommand{\K}[1]{\mathcal{K}(#1)}

\newcommand{\med}[1]{\mathrm{Median}(#1)}
% \newcommand{\mode}[1]{\mathrm{Mode}(#1)}
\newcommand{\ev}[1]{\mathbb{E}\!\left(#1\right)}
\newcommand{\var}[1]{\mathrm{Var}\!\left(#1\right)}
\newcommand{\ind}[1]{\mathds{1}_{#1}}
\newcommand{\pro}[1]{\mathbb{P}\!\left(#1\right)}